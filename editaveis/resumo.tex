\begin{resumo}
Este trabalho consiste no levantamento dos métodos de Aprendizagem de Máquina e na discussão acerca da necessidade de uso de Aprendizagem Incremental. Todo o fluxo do Aprendizado de Máquina é exposto, bem como a necessidade da utilização de Aprendizagem Incremental, com ênfase em algumas de suas peculiaridades. O método não-supervisionado de agrupamento incremental TASOM é introduzido. Experimentos são realizados visando comparar a performance deste método com o SOM, seu método de origem. Um estudo de caso baseado em um problema real do Ministério da Agricultura, Pecuária e Abastecimento é apresentado, e objetivos para o desenvolvimento de uma solução são descritos. A base de dados de plantas do tipo Iris também é utilizada para validar o TASOM.


\vspace{\onelineskip}
    
 \noindent
 \textbf{Palavras-chave}: Aprendizagem de Máquina, Aprendizagem Incremental, Agrupamento - Clusterização.

\end{resumo}
