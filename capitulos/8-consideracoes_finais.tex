\chapter{Considerações Finais}
Na introdução deste documento foram levantados os seguintes objetivos: Pesquisa e desenvolvimento de um método de Aprendizagem de Máquina do tipo não supervisionado, de agrupamento e incremental. Um levantamento teórico foi realizado visando compreender os processos de aprendizagem de máquina, os conceitos de aprendizagem incremental e o conhecimento de métodos de agrupamento incremental. As pesquisas levaram a escolha do método TASOM como um algoritmo não-supervisionado de agrupamento incremental. Este método foi escolhido por ser uma variação do SOM que atende aos requisitos do trabalho e por ter um mapa de saída comparável com o SOM. Desta forma é possível avaliar o desempenho dos dois em contexto de aplicação similar.

Os dois algoritmos foram implementados e um protótipo foi criado para a realização dos experimentos. Utilizando um conjunto de dados sintéticos, que incorporava diferentes possibilidades de agrupamentos, e considerando-se uma utilização em ambiente estático, o TASOM obteve desempenho similar ao SOM. Em um contexto dinâmico, onde o SOM clássico não funciona, o TASOM também obteve êxito, tendo erro muito baixo. Os parâmetros de análise nestes experimentos foram puramente matemáticos, extraindo assim a essência da ideologia de criação dos métodos.

Para validar os métodos em um contexto de utilização real, dois conjuntos de dados foram utilizados. Em todos eles tentou-se observar se os algoritmos seriam capazes de separar seus agrupamentos tendo em vista rótulos pré-estabelecidos. Está não é a aplicação para a qual estes sistemas foram criados, pois eles partem do pré-suposto não supervisionado e agrupam os dados com base em similaridade matemática. Mesmo assim, foi observado no experimento com dados de plantas Iris, que os métodos geraram agrupamentos interessantes tendo em vista a classe das amostras. Neste caso, o TASOM apresentou melhor desempenho que o SOM. 

A utilização de métodos de agrupamento foi feita em problemas que possuem rótulo para servir como um processo auxiliador aos métodos supervisionados. Através da observação do agrupamento natural das entradas é possível observar comportamentos importantes para entender os dados. Se os agrupamentos separam o espaço da mesma forma que as classes separam as amostras, este é um sistema que possui separabilidade linear e baixo ruído, como foi demonstrado no experimento com a base de dados de Iris. O resultado contrário indica a presença de ruído ou a existência de dados linearmente inseparáveis, como foi demonstrado do experimento do MAPA. Estas informações podem guiar novas etapas no ciclo de aprendizagem de máquina, como um reavaliação dos processo de coleta ou uma melhor engenharia de características. Ainda é possível tomar decisões a respeito dos métodos a serem utilizados, pois a morfologia do espaço a ser analisado se torna conhecida. 

Os resultados encontrados neste trabalho apontam para o TASOM como uma boa alternativa ao SOM em ambientes estáticos, semi-estáticos e dinâmicos. Os objetivos propostos foram alcançados, tendo-se um método não-supervisionado de agrupamento incremental implementado e testado.

\section{Futuros Trabalhos}
Como pode-se observar nos experimentos do capítulo 6, o resultado do TASOM varia muito por causa da inicialização randômica dos pesos. Esta sensibilidade pode influenciar no desempenho. Portanto, é necessário pesquisar novos métodos de inicialização dos pesos iniciais da rede que não introduzam viés nos agrupamentos e que otimizem a performance do algoritmo.

Como pôde ser visto também, os parâmetros livres do TASOM não são de trivial inicialização. Uma possibilidade de melhoria é criar mecanismos de parametrização automática que otimizem a performance da rede.    

   