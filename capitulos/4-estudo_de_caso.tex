\chapter{Estudo de Caso}
O estudo de caso que será usado para validar os métodos de Aprendizagem de Máquina com Aprendizagem Incremental é baseado em um problema real do Ministério da Agricultura, Pecuária e Abastecimento (MAPA). Este ministério é responsável  "pela gestão das políticas públicas de estímulo à agropecuária, pelo fomento do agronegócio e pela regulação e normatização de serviços vinculados ao setor" \cite{mapa}. Um dos objetivos do MAPA é garantir a segurança alimentar do povo brasileiro além de suportar a produção de exportação, garantindo o sucesso dos produtos brasileiros no mercado internacional.

Uma das atribuições do Ministério da Agricultura é velar pela segurança dos produtos agropecuários que entram e saem do país. Para isto existe o Sistema de Vigilância Agropecuária Internacional (Vigiagro) que é vinculado a Secretaria de Defesa Agropecuária. O vigiagro "atua na inspeção e fiscalização do trânsito internacional de vegetais, seus produtos e subprodutos. A fiscalização é feita nos portos, aeroportos internacionais, postos de fronteira e aduanas especiais" \cite{vigiagro}.

Neste contexto o vigiagro tem a atribuição fiscalizar todas as importações e exportações de produtos agropecuários do Brasil. Cada importação/exportação gera um requerimento, um documento onde o exportador/importador informa que realizará uma transação comercial internacional de um produto que está sobre vigilância do Vigiagro. Todos os requerimentos são inseridos no Sistema de Informações Gerenciais (SIGVIG). Os funcionários do Ministério então verificam vários itens, alguns de cunho documental e outros de cunho físico do produto comercializado. Caso haja alguma inconformidade o fiscal que avalia o requerimento gera um Termo de Ocorrência (TO) explicando a natureza e razão do erro encontrado, seja documental ou físico.

O problema de todo este processo é que no Brasil a quantidade de requerimentos gerada é enorme. O processo de fiscalização é extremamente trabalhoso e demanda muito tempo dos fiscais. Propõe-se utilizar métodos de Aprendizado de Máquina para criar um sistema que auxilie na fiscalização, diminuindo o tempo de processamento dos requerimentos e aumentando a qualidade da fiscalização. Este sistema deve ser capaz de analisar todos os requerimentos feitos associados com o resultado da fiscalização, deferimento ou indeferimento do processo. Através desta análise deve ser possível classificar os exportadores/importadores em três categorias: Alta Conformidade, Média Conformidade e Baixa Conformidade. Através destes três rótulos é possível criar políticas de fiscalização que foquem os esforços nos comerciantes que geralmente possuem baixa conformidade em seus requerimentos, fazendo com que os comerciantes que sempre possuem conformidade tenham seus processos agilizados e os que possuem baixa conformidade tenham seus requerimentos avaliados cuidadosamente. 

