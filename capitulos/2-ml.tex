\chapter{Machine Learning}

Aprendizado de Máquina, ou comumente conhecido pelo termo em inglês Machine Learning, é um novo campo do conhecimento que é composto pela intersecção entre a ciência da computação, estatística e matemática. Tem forte relacionamento  com a inteligência artifical, pois se assemelham em objetivos e técnicas. Aprendizado de Máquina se preocupa em habilitar sistemas computacionais a aprender ou desempenhar determinada função sem ser diretamente programdos para aquela tarefa. De acordo com Arthur Lee Samuel , cientista pioneiro do campo de inteligência artificial, Machine Learning é: "Field of study that gives computers the ability to learn without being explicitly programmed"  \citeonline{big2013}. 

Geralmente programas de computador são feitos pra resolver tarefas computacionais de forma explícita, isto é, o programador conhece plenamente o problema que deve ser atacado e comanda o computador a executar passos que correspondem as etapas necessárias para a resolução do problema. O humano por trás do algoritmo deve conhecer todos os relaciomentos do problema e ser capaz de identificar mentalmente todos os processos envolvidos na resolução. Isso nem sempre é possível, pois existem problemas tão complexos que tornam a abstração de sua resolução impossível para a capacidade humana, são muitos relaciomentos escondidos que tornam a tarefa de programar explicitamente inviável, programar explicitmente requer conhecer todos os passos da resolução do problema.

Para esse tipo de trabalho é que o Aprendizado de Máquina nasceu para solucionar, em vez de programar cada etapa de forma exaustiva a ideia é ensinar o computador a aprender sozinho as etapas de resolução do problema. Isso geralmente se dá quando é aprensentado ao computador um conjunto de dados de determinado contexto, nesses dados estão contidos todos os relaciomentos implícitos que compõem a solução, a tarefa do computador é extrair esses relaciomentos para ser capaz de generalizar uma solução. Por isso muitas definições de AM estão relacionadas aos dados, programas desse tipo são capazes de aprender, crescer e se modificar quando apresentados a novos conjuntos de dados.


Contexto de ML
Justificativa
Processos de ML

