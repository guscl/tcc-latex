\chapter{Aprendizado de Máquina}

Com o advento do crescimento do uso de Tecnologia da Informação na sociedade moderna, uma enorme quantidade de dados vem sendo gerada diariamente. Big Data é o termo em inglês usado para descrever o fênomeno deste grande volume de informações que a humanidade gera atualmente. Citam-se como exemplos de fontes dessas informações: redes sociais, registros de transações e dados de sensores. Estes conjuntos de elementos tem a caracteŕistica de serem desestruturados e crescerem muito rapidamente com o tempo. De acordo com o relatório produzido em 2013 pela União Europeia, \citeonline{eu2013}, aproximadamente 90\% de todos os dados gerados pela humanidade em 2013 foram feitos nos dois anos anteriores. Ainda afirma-se neste relatório que o desafio moderno não esta na capacidade de armazenamento dos registros, mas sim em usá-los de forma prática e eficiente, estima-se que apenas um quinto desses registros sejam de origem numérica.  

Aprendizado de Máquina, ou comumente conhecido pelo termo em inglês Machine Learning, é um novo campo do conhecimento que é composto pela intersecção entre a ciência da computação, estatística e matemática. Tem forte relacionamento  com a inteligência artificial, pois se assemelham em objetivos e técnicas. Aprendizado de Máquina se preocupa em habilitar sistemas computacionais a aprender ou desempenhar determinada função sem ser diretamente programdos para aquela tarefa. De acordo com Arthur Lee Samuel, cientista pioneiro do campo de inteligência artificial, Aprendizado de Máquina é: "Field of study that gives computers the ability to learn without being explicitly programmed" apud \citeonline{big2013}. Sistemas computacionais baseados em métodos AM são necessários para solucionar problemas altamente técnicos e especializados, algumas das tarefas que demandam o uso desses métodos sçao:aprendizado, raciocínio, planejamento, tomada de decisões, classificações, predições, dentre outros. 

Geralmente programas de computador são feitos pra resolver tarefas computacionais de forma explícita, isto é, o programador conhece plenamente o problema que deve ser atacado e comanda o computador a executar passos que correspondem as etapas necessárias para a resolução do problema. O humano por trás do algoritmo deve conhecer todos os relaciomentos do problema e ser capaz de identificar mentalmente todos os processos envolvidos na resolução. Isso nem sempre é possível, pois existem problemas tão complexos que tornam a abstração de sua resolução impossível para a capacidade humana, são muitos relaciomentos escondidos que tornam a tarefa de programar explicitamente inviável, pois isso requer conhecer todos os passos da resolução do problema.

Para esse tipo de trabalho é que o Aprendizado de Máquina nasceu para solucionar, em vez de programar cada etapa de forma exaustiva a ideia é ensinar o computador a aprender sozinho as etapas de resolução do problema. Isso geralmente se dá quando é aprensentado ao computador um conjunto de dados de determinado contexto, nesses dados estão contidos todos os relaciomentos implícitos que compõem a solução, a tarefa do computador é extrair esses relaciomentos para ser capaz de generalizar uma solução. Por isso muitas definições de AM estão relacionadas aos dados, programas desse tipo são capazes de aprender, crescer e se modificar quando apresentados a novos conjuntos de dados.





Contexto de ML
Justificativa (Número, Big Data, Aprendizado incremental)
Processos de ML (Tipos de pré-processamento)

