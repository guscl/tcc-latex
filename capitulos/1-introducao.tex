\chapter{Introdução}


Com o advento do crescimento do uso de Tecnologia da Informação na sociedade moderna, uma enorme quantidade de dados vem sendo gerada diariamente. Big Data é o termo em Inglês usado para descrever o fenômeno deste grande volume de informações que a sociedade gera atualmente. Citam-se como exemplos de fontes destas informações: redes sociais, registros de transações e dados de sensores. Estes conjuntos de elementos têm a característica de serem desestruturados, crescerem muito rapidamente com o tempo e terem alta complexidade.

De acordo com o relatório produzido em 2013 pela União Europeia, UE, aproximadamente 90\% de todos os dados gerados pela humanidade em 2013 surgiram nos dois anos anteriores \cite{eu2013}. Ainda afirma-se neste relatório que o desafio moderno não está na capacidade de armazenamento dos registros, mas, sim, em usá-los de forma prática e eficiente. Estima-se que apenas um quinto desses registros sejam de origem numérica. De acordo com \citeonline{lopez2011}, em um trabalho que estimou a capacidade tecnológica do mundo de armazenar, comunicar e processar informação: 

\begin{enumerate}
\item As telecomunicações foram dominadas pelas tecnologias digitais desde 1999, correspondendo a 99.9\% no ano de 2007. 
\item A maior parte da informação tecnológica da humanidade está em formato digital desde os anos 2000, correspondendo a 94\% no ano de 2007.
\item A capacidade de armazenamento de informação tecnológica \textit{per-capita} vem dobrando a cada 40 meses desde a década de 1980.
\end{enumerate}
   

Considerando-se os desafios apresentados, Aprendizado de Máquina (AM), ou comumente conhecido pelo termo em Inglês, Machine Learning, é um novo campo do conhecimento que possui elementos  interseccionados entre: Ciência da Computação, Estatística, Teoria da Informação, Filosofia, Biologia, Ciência Cognitiva, Complexidade Computacional e Teoria de Controle \cite{mitch1997}. Ela tem forte relacionamento  com a Inteligência Artificial, pois se assemelham em objetivos e técnicas. 

Aprendizado de Máquina se preocupa em habilitar sistemas computacionais a aprender ou desempenhar determinada função sem ser diretamente programados para aquela tarefa. De acordo com Arthur Lee Samuel, cientista pioneiro do campo de Inteligência Artificial, Aprendizado de Máquina é: "Campo de estudo que habilita computadores a aprender sem ser explicitamente programados" (A.L.Samuel apud \citeonline{big2013}).

Sistemas computacionais baseados em modelagens de AM são necessários para solucionar problemas altamente técnicos e especializados. Algumas das tarefas que demandam o uso destes métodos são: aprendizagem, raciocínio, planejamento, tomada de decisões, classificações e predições. 

Geralmente sistemas computacionais são feitos para resolver tarefas de forma explícita, isto é, o programador conhece plenamente o problema que deve ser atacado e comanda o computador a executar passos que correspondem às etapas necessárias para a resolução do problema.
Os programadores usam algoritmos para habilitar sistemas computacionais a realizar diversas tarefas. Um algoritmo é uma coleção bem ordenada de operações computacionais claras e efetivas que, quando executadas, produzem um resultado e param em uma quantidade finita de tempo \cite{schn1995}. 

O responsável pela programação do algoritmo deve conhecer todos os relacionamentos do problema e ser capaz de identificar mentalmente todos os processos envolvidos na resolução. Isto nem sempre é possível, pois existem problemas tão complexos que tornam a abstração de sua resolução impossível para a capacidade humana. São muitos relacionamentos escondidos que tornam a tarefa de programar explicitamente inviável, pois isto requer conhecer todos os passos da resolução do problema.

Para este tipo de trabalho é que o Aprendizado de Máquina nasceu para solucionar. Em vez de programar cada etapa de forma exaustiva, o objetivo é ensinar o computador a aprender sozinho as etapas de resolução do problema. Isto geralmente se dá quando é apresentado ao computador um conjunto de dados de determinado contexto. Nestes dados estão contidos todos os relacionamentos implícitos que compõem a solução. 

A tarefa do computador é extrair estes relacionamentos para ser capaz de generalizar uma solução. Sistemas de AM são preparados para aprender, crescer e se modificar quando apresentados a novos conjuntos de dados, por isto muitas definições de AM estão relacionadas aos dados.

Financeiramente, Aprendizado de Máquina é um campo que vem obtendo ascendente crescimento de investimento. Citando novamente o relatório da UE, \citeonline{eu2013}, somente na Europa estima-se que o mercado de sistemas especialistas gerou um total de 700 milhões de euros em 2013. As previsões são de que este número chegue a 27 bilhões de euros no ano de 2015. De acordo com as suposições dos autores deste relatório, os trabalhos na área de AM irão impulsionar uma grande demanda por profissionais altamente capacitados, além de incentivar pesquisas em diversas áreas do conhecimento. Isto tudo gerando expansões e investimentos, tanto nos Estados Unidos, quanto nos países que compõem o BRIC (Brasil, Rússia, Índia e China). 

O objetivo deste trabalho é a pesquisa e desenvolvimento de um método de Aprendizagem de Máquina do tipo não supervisionado, de agrupamento e incremental. A pesquisa serve para encontrar modelos adequados ao Aprendizado Incremental e o desenvolvimento visa validar estes modelos de forma comparativa. Neste trabalho é feito um levantamento teórico sobre o processo de Aprendizagem de Máquina e sobre Aprendizagem Incremental. Dois algoritmos não supervisionados e de agrupamento são escolhidos, um clássico que não possui aprendizagem incremental e outra mais recente que incorpora este conceito incremental. Os dois algoritmos são visando validar o desempenho do novo com os resultados do clássico. Um estudo de caso real é apresentado para testar o algoritmo incremental em contexto prático.

O desenvolvimento deste documento nos próximos capítulos é dividido em 7 partes. A primeira  trata de Aprendizagem de Máquina e nela é definido o que é AM e quais são as fases de seu desenvolvimento. O segundo capítulo trata de Aprendizagem Incremental. A importância desse tipo de aprendizagem é demonstrada e algumas discussões são feitas, comparando a visão de alguns pesquisadores com a visão da comunidade de AM. Na terceira parte, é discutido sobre o tipo de problema de AM da classe agrupamento, o algoritmo SOM (Self Organizing Map) é apresentado e algumas versões deste algoritmo na forma incremental são apresentadas. Logo após o algoritmo TASOM (Time Adaptative Self Organizing Map) é detalhado. Na quinta parte é apresentada a metologia de comparação entre o SOM e o TASOM, bem como os resultados dessa comparação. A sexta parte contém um estudo de caso que se baseia em um problema real do Ministério da Agricultura Pecuária e Abastecimento, MAPA. A última parte contém as considerações finais e propostas para a continuação do trabalho.





