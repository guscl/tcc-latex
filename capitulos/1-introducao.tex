\chapter{Introdução}


Com o advento do crescimento do uso de Tecnologia da Informação na sociedade moderna, uma enorme quantidade de dados vem sendo gerada diariamente. Big Data é o termo em inglês usado para descrever o fenômeno deste grande volume de informações que a sociedade gera atualmente. Citam-se como exemplos de fontes dessas informações: redes sociais, registros de transações e dados de sensores. Estes conjuntos de elementos tem a característica de serem desestruturados, crescerem muito rapidamente com o tempo e terem alta complexidade. De acordo com o relatório produzido em 2013 pela União Europeia, UE, aproximadamente 90\% de todos os dados gerados pela humanidade em 2013 foram feitos nos dois anos anteriores \cite{eu2013}. Ainda afirma-se neste relatório que o desafio moderno não esta na capacidade de armazenamento dos registros, mas sim em usá-los de forma prática e eficiente, estima-se que apenas um quinto desses registros sejam de origem numérica. De acordo com \cite{lopez2011} em um trabalho que estimou a capacidade tecnológica do mundo de armazenar, comunicar e processar informação: 
\begin{enumerate}
\item As telecomunicações foram dominadas pelas tecnologias digitais desde 1999, correspondendo a 99.9\% no ano de 2007. 
\item A maior parte da informação tecnológica da humanidade está em formato digital desde os anos 2000, correspondendo a 94\% no ano de 2007.
\item A capacidade de armazenamento de informação tecnológica per-capita vem dobrando a cada 40 meses desde a década de 1980.
\end{enumerate}
   

Considerando-se os desafios apresentados, Aprendizado de Máquina, ou comumente conhecido pelo termo em inglês Machine Learning, é um novo campo do conhecimento que é possui elementos  interseccionados entre: Ciência da Computação, Estatística, Teoria da Informação, Filosofia, Biologia, Ciência Cognitiva, Complexidade Computacional e Teoria de Controle \cite{mitch1997}. Tem forte relacionamento  com a Inteligência Artificial, pois se assemelham em objetivos e técnicas. Aprendizado de Máquina se preocupa em habilitar sistemas computacionais a aprender ou desempenhar determinada função sem ser diretamente programados para aquela tarefa. De acordo com Arthur Lee Samuel, cientista pioneiro do campo de inteligência artificial, Aprendizado de Máquina é: "Campo de estudo que habilita computadores a aprender sem ser explicitamente programados" (A.L.Samuel apud \citeonline{big2013}). Sistemas computacionais baseados em modelagens de AM são necessários para solucionar problemas altamente técnicos e especializados, algumas das tarefas que demandam o uso desses métodos são: aprendizado, raciocínio, planejamento, tomada de decisões, classificações, predições. 

Geralmente sistemas computacionais são feitos para resolver tarefas computacionais de forma explícita, isto é, o programador conhece plenamente o problema que deve ser atacado e comanda o computador a executar passos que correspondem as etapas necessárias para a resolução do problema. Programadores usam algoritmos para habilitar sistemas computacionais a realizar diversas tarefas, um algoritmo é uma coleção bem ordenada de operações computacionais claras e efetivas que quando executadas produzem um resulto e param em uma quantidade finita de tempo \cite{schn1995}. O humano por trás do algoritmo deve conhecer todos os relacionamentos do problema e ser capaz de identificar mentalmente todos os processos envolvidos na resolução. Isso nem sempre é possível, pois existem problemas tão complexos que tornam a abstração de sua resolução impossível para a capacidade humana. São muitos relacionamentos escondidos que tornam a tarefa de programar explicitamente inviável, pois isso requer conhecer todos os passos da resolução do problema.

Para esse tipo de trabalho é que o Aprendizado de Máquina nasceu para solucionar, em vez de programar cada etapa de forma exaustiva a ideia é ensinar o computador a aprender sozinho as etapas de resolução do problema. Isso geralmente se dá quando é apresentado ao computador um conjunto de dados de determinado contexto, nesses dados estão contidos todos os relacionamentos implícitos que compõem a solução, a tarefa do computador é extrair esses relacionamentos para ser capaz de generalizar uma solução. Sistemas de AM são capazes de aprender, crescer e se modificar quando apresentados a novos conjuntos de dados, por isso muitas definições de AM estão relacionadas aos dados.

Financeiramente, Aprendizado de Máquina é um campo que vem obtendo ascendente crescimento de investimento. Citando novamente o relatório da UE, \citeonline{eu2013}, somente na Europa estima-se que o mercado de sistemas especialistas gerou um total de 700 milhões de euros em 2013 e as previsões são de que esse número chegue a 27 bilhões de euros no ano de 2015. De acordo com as suposições dos autores deste relatório, os trabalhos na área de AM irão impulsionar uma grande demanda por profissionais altamente capacitados, além de incentivar pesquisas em diversas áreas do conhecimento, como na forma como compreende-se o cérebro humano. Isto tudo gerando expansões e investimentos, tanto nos Estados Unidos, quanto nos países que compõem o BRIC (Brasil, Rússia, Índia e China). 

O objetivo deste trabalho é a pesquisa e desenvolvimento de um método de Aprendizagem de Máquina com Aprendizado Incremental. A pesquisa serve para encontrar modelos adequados ao Aprendizado Incremental e o desenvolvimento visa validar estes modelos de forma comparativa. Neste trabalho é feito um levantamento teórico sobre o processo de Aprendizagem de Máquina e sobre Aprendizagem Incremental. Um estudo de caso real é descrito para servir de apresentação a parte de desenvolvimento do trabalho. Esta etapa serve para lançar os conhecimentos que servirão de base para as fases futuras do trabalho: Escolha de algoritmos de Aprendizagem de Máquina com característica de Aprendizagem Incremental e desenvolvimento destes modelos visando uma análise comparativa.

O desenvolvimento deste documento é dividido em 4 capítulos. O primeiro capítulo trata de Aprendizagem de Máquina. Nele é definido o que é AM e quais são as fases de seu desenvolvimento. O segundo capítulo trata de Aprendizagem Incremental, a importância desse tipo de aprendizagem é demonstrada e algumas discussões são feitas comparando a visão de alguns pesquisadores e a visão da comunidade de AM. No terceiro capítulo é apresentado um estudo de caso que se baseia em um problema real do Ministério da Agricultura Pecuária e Abastecimento, MAPA. O último capítulo contém as considerações finais e propostas para a continuação do trabalho.





