\chapter{Estudo de Caso}
O estudo de caso que será usado para validar os métodos de Aprendizagem de Máquina com Aprendizagem Incremental é baseado em um problema real do Ministério da Agricultura, Pecuária e Abastecimento (MAPA). Este ministério é responsável  "pela gestão das políticas públicas de estímulo à agropecuária, pelo fomento do agronegócio e pela regulação e normatização de serviços vinculados ao setor" \cite{mapa}. Um dos objetivos do MAPA é garantir a segurança alimentar do povo brasileiro além de suportar a produção de exportação, garantindo o sucesso dos produtos brasileiros no mercado internacional.

Para resolver o problema apresentado no estudo de caso do capítulo anterior é necessário organizar as entradas, para que se possa compreender qual o relacionamento entre as características de cada amostra com o resultado da fiscalização feita pelo MAPA. Se for possível encontrar informações que são determinantes para o resultado da análise feita pelo laboratório, será possível aprimorar os critérios de amostragem, que hoje são randômicos, melhorando a eficácia da fiscalização de resíduos contaminantes em aves e suínos consumidos no Brasil.

Para um estudo inicial é interessante observar qual é o agrupamento natural dos dados. Alterando a combinação das variáveis de entrada e agrupando as amostras, é possível analisar cada agrupamento e medir a quantidade de violações em cada conjunto. Dessa forma, pode ser possível constatar quais as características são determinantes, por analisar a morfologia de cada agrupamento em cada combinação de características. Este tipo de problema cai na categoria de agrupamento.
















