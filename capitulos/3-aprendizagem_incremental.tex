\chapter{Aprendizado Incremental}


Um outro desafio que emerge da grande quantidade de informações que são geradas é, o tempo de validade dos modelos de AM. Atualmente, a maioria dos modelos de aprendizagem possuem uma etapa de treino que acontece antes da etapa de uso efetivo dele. Esta etapa serve para que o sistema aprenda as características do espaço do problema, para depois ser capaz de atuar nos dados que vem sendo gerados em tempo real. 
O processo de construção de um sistema especialista efetivo geralmente demanda muito tempo e trabalho. É possível que estes sistemas se tornem menos precisos ou ineficazes em pouco tempo, pois a geração de dados no contexto do problema pode ser muito rápida e novas características podem emergir dos registros mais recentes. Se estes modelos de AM só conseguem aprender na etapa de treino, há um problema, pois eles não serão capazes de lidar com conceitos de aprendizagem que apareceram somente nas informações mais novas. 

Para aumentar o tempo de validade dos sistemas de Aprendizagem Máquina é possível usar Aprendizagem Incremental, isto consiste em capacitar modelos de AM a aprender continuamente, conforme novas entradas chegam ao sistema. Desta forma, mesmo que uma característica surja somente nos registros mais recentes, o sistema será capaz de aprender novamente e atuar de forma efetiva neste novo contexto. É importante ressaltar que o agente especialista deve ser capaz de guardar conhecimento na forma de inteligência no sistema, isto é, ele não tem acesso aos dados que já passaram por ele, mas tem em sua morfologia o conhecimento necessário que foi aprendido quando estes dados passaram por ele. Essa necessidade vem do fato de que é muito custoso armazenar todos os dados que já passaram pelo sistema, pois o volume de informação neste contexto é enorme. \cite{incremental2011}.

