\chapter{Aprendizado Incremental}


Um outro desafio que emerge da grande quantidade de informações que são geradas é, o tempo de validade dos modelos de AM. Atualmente, a maioria dos modelos de aprendizagem possuem uma etapa de treino que acontece antes da etapa de uso efetivo dele. Esta etapa serve para que o sistema aprenda as características do espaço do problema, para depois ser capaz de atuar nos dados que vem sendo gerados em tempo real. 
O processo de construção de um sistema especialista efetivo geralmente demanda muito tempo e trabalho. É possível que estes sistemas se tornem menos precisos ou ineficazes em pouco tempo, pois a geração de dados no contexto do problema pode ser muito rápida e novas características podem emergir dos registros mais recentes. Se estes modelos de AM só conseguem aprender na etapa de treino, há um problema, pois eles não serão capazes de lidar com conceitos de aprendizagem que apareceram somente nas informações mais novas. 

Para aumentar o tempo de validade dos sistemas de Aprendizagem Máquina é possível usar Aprendizagem Incremental, isto consiste em capacitar modelos de AM a aprender continuamente, conforme novas entradas chegam ao sistema. Desta forma, mesmo que uma característica surja somente nos registros mais recentes, o sistema será capaz de aprender novamente e atuar de forma efetiva neste novo contexto. É importante ressaltar que o agente especialista deve ser capaz de guardar conhecimento na forma de inteligência no sistema, isto é, ele não tem acesso aos dados que já passaram por ele, mas tem em sua morfologia o conhecimento necessário que foi aprendido quando estes dados passaram por ele. Essa necessidade vem do fato de que é muito custoso armazenar todos os dados que já passaram pelo sistema, pois o volume de informação neste contexto é enorme \cite{incremental2011}.

A tendência da necessidade de Aprendizado Incremental é clara, como foi dito na introdução deste trabalho, o volume de informações gerado Humanidade tem crescido de forma acelerada e faz-se necessário processar toda esta informação de forma eficaz.As fontes modernas de dados não só são altamente dinâmicas como produzem informação em uma velocidade acelerada. As principais características de um contexto que requere aprendizagem incremental comparado as abordagens tradicionais são \cite{batch2013}:

\begin{enumerate}
\item Necessidade de realizar previsões a qualquer momento.
\item A base de dados evolui constantemente com o tempo.
\item É esperado uma entrada infinita de dados, porém os recursos computacionais são finitos
\end{enumerate}

Para entender aprendizado incremental é necessário compreender a diferença entre duas formar distintas que modelos de AM podem ter, aprendizado online e por fornada. Aprendizado por fornada é o tipo mais comum de algoritmos que são usados atualmente, eles possuem duas fases claramente distintas, treino e uso. Na fase de treino o modelo é capaz de aprender os relacionamentos que estão ocultos nos dados e na fase de uso não há aprendizado, há somente predições. Os algoritmos que possuem aprendizado online não possuem essa distinção em suas etapas, eles são capazes de aprender novos conceitos a todo o tempo e o fazem junto com as predições usuais. É esperado que depois da fase de treinamento os algoritmos que possuem aprendizado por fornada tenham criado em suas estruturas internas uma função hipótese que seja capaz de generalizar corretamente em qualquer conjunto de dados daquele contexto. No aprendizado online não se espera que uma hipótese seja criada para aquele contexto de informações, o algoritmo trata todas as entradas de maneira única, uma por uma, sempre aglomerando os conhecimentos desta entrada \cite{from2009}. 

Existe um terceiro tipo de aprendizagem que deriva da aprendizagem por fornada e é capaz de resolver os problemas acima citados comparando-se com a aprendizagem online, é a abordagem fornada-incremental.  

 

